\documentclass[twocolumn]{scrartcl}
\usepackage{enumitem}
\usepackage{wrapfig}
\usepackage{todonotes}
\usepackage{marginnote}
%opening
\title{Conlang questions from Describing Morphosyntax}
\author{Dicewitch}

\begin{document}

\maketitle
\onecolumn
\tableofcontents
\pagebreak
\twocolumn
\section{Demographic and ethnographic information}
\begin{enumerate}
\subsection{Name of the language}
\item What is the language known as to outsiders?
\item What term do the people use to distinguish themselves from other language groups?
\item What is the origin of these terms?
\subsection{Ethnology}
\item What is the dominant economic activity of the people?
\item Briefly describe the ecosystem, material culture, and cosmology.
\subsection{Demography}
\item Where is the language spoken?
\item How are the people distributed in this area?
\item Are there other language groups inhabiting the same area?
\item What is the nature of the interaction with these language groups? Economic? Social? Friendly? Belligerent?

\item In social/economic interactions with other groups, which groups are dominant and which are marginalized?  How so?
\subsection{Genetic affiliation}
\item What language family does this language belong to?
\item What are its closest relatives?
\subsection{The sociolinguistic situation}
\subsubsection{Multilingualism and language attitudes}
\item What percentage of people are monolingual? (Treat men and women separately)
\item What language(s) are people multilingual in, and to what degree?
\item What is the attitude of the speakers of this language, as opposed to other languages they may know?
\subsubsection{Contexts of use and language use}
\item In what contexts are multilingual individuals likely to use the language described in this sketch?  When do they use other languages?
\subsubsection{Viability}
\item Are children learning the language as their first language?  If so, how long do they remain monolingual?
\item What pressures are there on young people to learn another language or reject their own language?  How strong are these pressures?
\subsubsection{Loan words}
\item Does the lexicon of this language contain many words from the other languages? If so, in what semantic domains do these tend to occur?
\subsection{Dialects}
\item Is there significant dialect variation? What kinds of differences distinguish the dialects?
\item What dialect is represented in this sketch?
\end{enumerate}
\section{Morphological Typology}
\subsection{Historical background and definitions}
\begin{enumerate}[resume]
\item Is the language dominantly isolating or polysynthetic?
\item If the language is at all polysynthetic, is it dominantly agglutinative or fusional?
\item Give examples of its dominant pattern and any secondary patterns.
\subsection{Morphological processes}
\item If the language is at all agglutinative, is it dominantly prefixing,suffixing, or neither?
\item Illustrate the major and secondary patterns (including infixation, stem modification, reduplication, suprasegmental modification, and suppletion)
\subsection{Head/dependent marking}
\item If the language is at all polysynthetic, is it dominantly "head-marking," "dependent-marking," or mixed?
\item Give some examples of each type of marking the language exhibits.
\section{Grammatical categories}
\subsection{Nouns}
\item What are the distributional properties of nouns? \marginpar{\textbf{Distributional properties:} How nouns appear in phrases, clauses and texts}
\item What are the structural properties of nouns?\marginpar{\textbf{Structural properties:} internal structure of the noun, ie. case, number, gender}
\item What are the major formally distinct subcategories of nouns? 
\subitem Proper names, possessable vs non-possessable, count vs mass, noun class (If you have a noun class system, put it in the noun chapter.\todo{add section ref})
\item What is the basic structure of the noun word (for polysynthetic languages) and/or noun phrase (for more isolating languages)?
\item Does the language have free pronouns and/or anaphoric clitics? \marginpar{\textbf{Anaphoric clitics} are pronouns that must be attached to another word.}
\item Give a chart of the free pronouns and or anaphoric clitics.
\subsection{Verbs}
\item What are the distributional properties of verbs? 
\item What are the structural properties of verbs?
\item What are the major subclasses of verbs?
\item Describe the order of various verbal operators within the verbal word or verb phrase? See the list below.\label{verbop}\todo{add section references}
\subitem verb agreement/concord
\subitem semantic role markers (applicatives)
\subitem valence increasing devices
\subitem valence decreasing operations
\subitem tense/aspect/mode (TAM)
\subitem evidentials
\subitem location and direction
\subitem speech act markers
\subitem verb(-phrase) negation
\subitem subordination/nominalization
\subitem switch-reference
\item Give charts of the various paradigms, e.g., person marking, tense/aspect/mode etc. Indicate major allomorphic variants.
\item Are directional and/or locational notions expressed in the verb or verb phrase at all?
\end{enumerate}
\textbf{Questions to answer for all verbal operations (see p\ref{verbop}):}
\begin{enumerate}[resume]
\item Is this operation obligatory, i.e., does one member of the paradigm have to occur in every finite verb or verb phrase?
\item Is it productive, i.e., can the operation be specified for all verb stems, and does it have the same meaning with each one? (Nothing is fully productive, but some operations are more productive than others)
\item Is this operation primarily coded \textbf{morphologically, analytically, or lexically?} Are there any exceptions to the general case?
\item Where in the verb phrase or verbal word is this operation likely to appear?  Can it occur in more than one place?
\subsection{Modifiers}
\item If you posit a morphosyntactic category of adjectives, give evidence for not grouping these forms with the verbs or nouns.
\item What characterizes a form as being an adjective in this language?
\item How can you characterize semantically the class of concepts coded by this formal category?
\item Do adjectives agree with their heads (e.g., in number, case and/or noun class)
\item What kind of system does the language employ for counting? Decimal, quintenary?
\item How high can a fluent native speaker count without resorting either to words from another language or to a generic word like \textit{many?} Exemplify the system up to this point.
\item Do numerals agree with their head nouns?
\subsection{Adverbs}\marginpar{\textbf{Types of adverbs} include \textbf{manner} (\textit{quickly, slowly, patiently, etc}), \textbf{time}, \textbf{direction /location}, and \textbf{evidential /epistemic} (indicates the source of information - hearsay, first-hand, second-hand, or conjecture)}
\item What characterizes a form as being an adverb in this language?  If you posit a distinct class of adverbs, argue for why these forms should not be treated as nouns, verbs, or adjectives.
\item For each kind of adverb listed in this section, list a few members of the type, and specify whether there are any restrictions relative to that type, e.g., where they can come in a clause, any morphemes common to the type, etc.
\item Are any of these classes of adverbs related to older complement-taking (matrix) verbs?
\section{Constituent order typology}
\textbf{General questions for all units of structure}
\item What is the neutral order of free elements in the unit?
\item Are there variations?
\item How do the variant orders function?

\textbf{Question specific to the main clause constituent order:}
\item What is the pragmatically neutral order of constituents (A/S, P, and V) in basic clauses of the language?
\subsection{Verb phrase}
\item Where do auxiliaries occur in relation to the semantically "main" verb?
\item Where do verb-phrase adverbs occur with respect to the verb and auxiliaries?
\subsection{Noun phrase}
\item Describe the order(s) of elements in the noun phrase.
\subsection{Adpositional phrases (prepositions and postpositions)}
\item Is the language dominantly prepositional or post-positional?  Give examples.
\item Do many adpositions come from nouns or verbs?
\subsection{Comparatives}
\item Does the language have one or more grammaticalized comparative constructions?
\item If so, what is the order of the standard, the marker, and the quality by which an item is compared to the standard?\marginpar{\textbf{The 3 elements of comparison:}

 (1) the known \textbf{standard} against which the subject is compared; (2) the \textbf{marker} that signals that the clause is comparative, and (3) the \textbf{quality} by which the subject is compared to the standard.}
\end{enumerate}
\end{document}
