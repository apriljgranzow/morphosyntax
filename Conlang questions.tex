\documentclass[]{scrartcl}
\usepackage{enumitem}
\usepackage{wrapfig}
\usepackage{tcolorbox}
\newenvironment{WrapText}[1][r]
  {\wrapfigure{#1}{0.5\textwidth}\tcolorbox}
  {\endtcolorbox\endwrapfigure}
%opening
\title{Conlang questions from \textit{Describing Morphosyntax}}
\author{Dicewitch}

\begin{document}

\maketitle
\tableofcontents
\pagebreak
\section{Demographic and ethnographic information}
\begin{enumerate}
\subsection{Name of the language}
\item What is the language known as to outsiders?
\item What term do the people use to distinguish themselves from other language groups?
\item What is the origin of these terms?
\subsection{Ethnology}
\item What is the dominant economic activity of the people?
\item Briefly describe the ecosystem, material culture, and cosmology.
\subsection{Demography}
\item Where is the language spoken?
\item How are the people distributed in this area?
\item Are there other language groups inhabiting the same area?
\item What is the nature of the interaction with these language groups? Economic? Social? Friendly? Belligerent?

\item In social/economic interactions with other groups, which groups are dominant and which are marginalized?  How so?
\subsection{Genetic affiliation}
\item What language family does this language belong to?
\item What are its closest relatives?
\subsection{The sociolinguistic situation}
\subsubsection{Multilingualism and language attitudes}
\item What percentage of people are monolingual? (Treat men and women separately)
\item What language(s) are people multilingual in, and to what degree?
\item What is the attitude of the speakers of this language, as opposed to other languages they may know?
\subsubsection{Contexts of use and language use}
\item In what contexts are multilingual individuals likely to use the language described in this sketch?  When do they use other languages?
\subsubsection{Viability}
\item Are children learning the language as their first language?  If so, how long do they remain monolingual?
\item What pressures are there on young people to learn another language or reject their own language?  How strong are these pressures?
\subsubsection{Loan words}
\item Does the lexicon of this language contain many words from the other languages? If so, in what semantic domains do these tend to occur?
\subsection{Dialects}
\item Is there significant dialect variation? What kinds of differences distinguish the dialects?
\item What dialect is represented in this sketch?
\end{enumerate}
\section{Morphological Typology}
\subsection{Historical background and definitions}
\begin{enumerate}[resume]
\item Is the language dominantly isolating or polysynthetic?
\item If the language is at all polysynthetic, is it dominantly agglutinative or fusional?
\item Give examples of its dominant pattern and any secondary patterns.
\end{enumerate}
\end{document}
